
\documentclass[12pt]{article}
\usepackage[a4paper, margin=1in]{geometry}
\usepackage{amsmath}
\usepackage{hyperref}
\usepackage{titlesec}
\titleformat{\section}{\large\bfseries}{\thesection}{1em}{}

\title{\textbf{Master’s Thesis Proposal} \\
A fast implementation of deep neural-network potentials for molecular dynamics simulations of binary alloy}
\author{}
\date{}

\begin{document}
\maketitle

\section*{Background and Motivation}
Machine-learned potentials, especially the Deep Potential (DP) models implemented in DeePMD-kit v2, offer a powerful alternative to classical force fields in molecular dynamics (MD) simulations. These models combine the high accuracy of ab initio methods DFT with the computational efficiency of classical MD, enabling realistic large-scale simulations. HALMD is a high-performance MD engine for Nvidia GPU.

This project extends DeePMD-trained potentials in HALMD to multi-component mixtures by incorporating type embeddings, allowing simulations of binary alloys and other mixed systems. Unlike previous work focused on single-species models (e.g., Cu), this thesis adds support for species-specific descriptors. Type embeddings introduce an additional one dimensional input to each atom type, enabling the model to distinguish between different elements during force and energy calculations.



\section*{Objectives}
The thesis is structured into two main parts:

\subsection*{Part 1: Integration of Deep Potential Models into HALMD}
\textbf{1. Model Parameter Extraction}

Develop standalone Python tool to directly extract and interpret the parameters (weights, biases, activation types, and cutoffs) from DeePMD-kit v2 trained models. This avoids limitations encountered when attempting to load the models using TensorFlow. The focus will be on models utilizing two-body embeddings (DeepPot-SE) in combination with type embeddings for multi-species systems.

The extraction pipeline will target publicly available .pb models trained using DeePMD-kit v2, including but not limited to:
\begin{itemize}
    \item Cu model (DeepPot-SE with type embedding) from AISSquare: \url{https://www.aissquare.com/models/detail?pageType=models&name=Cu_fcc_slabs}
    \item Ag-Au nanoalloy model trained with DeepPot-SE using .pb format: AgAu-nanoalloy-model \\
    \url{https://www.aissquare.com/models/detail?pageType=models&name=AgAu-nanoalloy-model}
    \item Additional .pb models with verified two-body embeddings and type embeddings from AISSquare and DeepModeling repositories, selected based on availability and documentation of the descriptor settings.
\end{itemize}

\textbf{2. Modular Potential Calculator in HALMD}

Design and implement a modular component in HALMD that:
\begin{itemize}
    \item Computes potentials from the extracted parameters,
    \item Supports two-body descriptor inference with type embedding integration for multi-component systems,
    \item Incorporates the additional input dimension introduced by type embeddings to distinguish atomic species,
    \item Handles periodic boundary conditions and neighbor lists consistently with DeePMD.
\end{itemize}

\textbf{3. Validation and Benchmarking}

Benchmark the accuracy of force and energy calculations by comparing HALMD outputs with reference results generated by DeePMD-kit for known systems.

\subsection*{Part 2: Material Properties from DeePMD-based MD Simulations}
\textbf{4. Thermodynamic Property Extraction}

Using trajectories from Part 1, compute:

\begin{itemize}
    \item Specific Heat Capacity (\(C_V\), \(C_P\)), which quantifies the system's ability to store thermal energy. Under the canonical (NVT) ensemble, it can be estimated using energy fluctuations as:
    \[
    C_V = \frac{1}{k_B T^2} \left( \langle E^2 \rangle - \langle E \rangle^2 \right)
    \]
    where:
    \begin{itemize}
        \item \(C_V\) is the specific heat at constant volume,
        \item \(k_B\) is the Boltzmann constant,
        \item \(T\) is the absolute temperature,
        \item \(E\) is the total energy (kinetic + potential),
        \item \(\langle E \rangle\) is the average energy over the simulation,
        \item \(\langle E^2 \rangle\) is the average of the squared energy.
    \end{itemize}
\end{itemize}

\begin{itemize}
    \item \textbf{Bulk Modulus}, which measures resistance to uniform compression:

    \begin{itemize}
        \item \textbf{Thermodynamic Definition}:
        \[
        K_T = -V \left( \frac{\partial P}{\partial V} \right)_T
        \]
        where:
        \begin{itemize}
            \item \( K_T \) is the isothermal bulk modulus,
            \item \( V \) is the system volume,
            \item \( P \) is the pressure,
            \item The derivative is taken at constant temperature \( T \).
        \end{itemize}

        \item \textbf{Microscopic Definition via Structure Factor (NVT Ensemble)}:
        \[
        K_T = \frac{\rho k_B T}{S(k \to 0)}
        \]
        where:
        \begin{itemize}
            \item \( \rho \) is the number density \( (N/V) \),
            \item \( k_B \) is the Boltzmann constant,
            \item \( T \) is the absolute temperature,
            \item \( S(k \to 0) \) is the static structure factor in the long-wavelength limit.
        \end{itemize}
    \end{itemize}
\end{itemize}


\textbf{5. Benchmarking with Reference Data}

To validate the computed thermodynamic properties, specific heat capacity and bulk modulus will be used as primary benchmarking targets. Simulation results will be compared against experimental or reference values from the NIST Chemistry WebBook or other public thermophysical property databases under matching conditions (e.g., temperature, pressure). Accuracy will be assessed using standard error metrics.

\section*{Methodology}
\begin{itemize}
    \item Study the DeePMD v2 architecture in detail, especially two-body embedding descriptors (DeepPot-SE), type embeddings, and ResNet-based fitting networks.
    \item Write a custom parser to extract model parameters directly from the \texttt{.pb} files.
    \item Extend the existing C++ module in HALMD developed for single-species systems (e.g., Cu) to support multicomponent systems by incorporating type embeddings. This includes processing the additional input dimension introduced by atom-type-specific vectors and adapting the descriptor and network evaluation accordingly.
    \item Run MD simulations for NVT ensembles with the integrated potential model.
    \item Post-process observable data using statistical mechanics formulas to calculate macroscopic thermodynamic properties.
\end{itemize}


\section*{Expected Outcome}
\begin{itemize}
    \item A modular DeePMD-compatible extension to HALMD, supporting two-body/type-embedded potentials.
    \item Accurate validation of machine-learned force fields within a high-performance MD engine.
    \item A complete framework for computing and benchmarking thermodynamic properties (e.g., heat capacity, bulk modulus).
    \item Potential open source contributions to the HALMD and DeePMD-kit communities.
\end{itemize}

\section*{Supervision and Timeline}
The thesis will be conducted over a period of five months. The work will be organized as follows:

\textbf{Month 1:}
\begin{itemize}
    \item Study DeePMD-kit v2 architecture and model internals
    \item Analyze \texttt{.pb} model structure and design parameter extractor
    \item Implement Python parser to extract model parameters
    % \item Set up HALMD build and plugin environment
\end{itemize}

\textbf{Month 2:}
\begin{itemize}
    \item Begin HALMD integration of potential inference module
    \item Use a Python prototype to reproduce the calculations of the DeepMD-trained model using the extracted parameters
    \item Complete HALMD integration for two-body/type-embedded potentials
\end{itemize}

\textbf{Month 3:}
\begin{itemize}
    
    \item Validate energy and force outputs against DeePMD reference data
    \item Run short MD simulations to test stability and consistency
\end{itemize}

\textbf{Month 4:}
\begin{itemize}
    \item Perform production-scale MD simulations 
    
    \item Post-process data to compute \(C_V\) and \(K_T\)
    \item Begin comparison against NIST or other reference databases
\end{itemize}

\textbf{Month 5:}
\begin{itemize}
    \item Finalize validation, benchmarking, and documentation
    \item Write the thesis and prepare for defense
    \item Deliver and package code and results for submission or release
\end{itemize}

\end{document}
