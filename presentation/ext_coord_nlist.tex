\documentclass[12pt,a4paper]{article}
\usepackage{amsmath}
\usepackage{amssymb}
\usepackage{geometry}
\usepackage{graphicx}
\usepackage{array}
\usepackage{booktabs}
\usepackage{caption}

\geometry{margin=1in}

\title{Summary of Coordinate Normalization, Extension, and Neighbor List Construction}
\author{}
\date{}

\begin{document}

\maketitle

\section*{1. Normalization}
\textbf{Goal:} Keep all atom positions inside the primary periodic box.

Each atom has position $\mathbf{r}_i$ in real space and a simulation box defined by lattice vectors $\mathbf{a}, \mathbf{b}, \mathbf{c}$.

We define the transformation matrix
\[
H = [\mathbf{a}, \mathbf{b}, \mathbf{c}]
\]
and convert each atom position into fractional coordinates:
\[
\mathbf{s}_i = H^{-1}\mathbf{r}_i
\]
The coordinates are then wrapped into the interval $[0,1)$:
\[
\mathbf{s}_i' = \mathbf{s}_i - \lfloor \mathbf{s}_i \rfloor
\]
and finally transformed back to real space:
\[
\mathbf{r}_i' = H \mathbf{s}_i'
\]

\textbf{Intuition:}  
Atoms that leave the simulation box due to periodic boundary conditions are wrapped back into the primary cell.  
This ensures all atoms lie within a single reference box for consistent neighbor searching.

\bigskip

\section*{2. Extension (Ghost Creation)}
\textbf{Goal:} Recreate periodic images around the simulation box to ensure that every atom has all neighbors within a cutoff radius $r_{\text{cut}}$, even across boundaries.

We first compute the face-to-face distances using the box vectors:
\[
n_{\text{buff}} = \left\lceil \frac{r_{\text{cut}}}{d_{\text{face}}} \right\rceil
\]

Then, for each periodic shift $(n_x, n_y, n_z)$ in the range $[-n_{\text{buff}}, +n_{\text{buff}}]$, we generate ghost atoms:
\[
\mathbf{r}_{i}^{(\text{ghost})} = \mathbf{r}_i' + n_x \mathbf{a} + n_y \mathbf{b} + n_z \mathbf{c}
\]

\textbf{Intuition:}  
This replicates the box in all three directions.  
Atoms near one boundary now see neighbors from the opposite boundary via these ghost atoms.

\bigskip

\section*{3. Neighbor List Construction}
\textbf{Goal:} Identify all atoms (real and ghost) within the cutoff distance $r_{\text{cut}}$ for each central atom.

For each atom $i$, compute pairwise distances:
\[
r_{ij} = \| \mathbf{r}_j^{(\text{extended})} - \mathbf{r}_i' \|
\]
We then:
\begin{itemize}
    \item Exclude self-interactions ($i = j$),
    \item Sort all other atoms by distance,
    \item Keep up to $\texttt{sel[type]}$ nearest neighbors for each atom type,
    \item Store the results in the neighbor list array $nlist[i, j]$.
\end{itemize}

\textbf{Intuition:}  
This defines the local atomic environment around each atom — which is later used by DeePMD to construct descriptors and compute energies.

\bigskip

\section*{Summary Table}

\begin{center}
\begin{tabular}{|>{\raggedright\arraybackslash}p{2.5cm}|>{\raggedright\arraybackslash}p{3.5cm}|>{\raggedright\arraybackslash}p{5cm}|>{\raggedright\arraybackslash}p{3cm}|}
\hline
\textbf{Step} & \textbf{Function} & \textbf{Key Equation / Logic} & \textbf{Purpose} \\
\hline
Normalization & \texttt{normalize\_coord()} & $\mathbf{r}_i' = H(H^{-1}\mathbf{r}_i - \lfloor H^{-1}\mathbf{r}_i \rfloor)$ & Wrap atoms into primary box \\
\hline
Extension & \texttt{extend\_coord\_with\_ghosts()} & $\mathbf{r}_i^{(\text{ghost})} = \mathbf{r}_i' + n_x\mathbf{a}+n_y\mathbf{b}+n_z\mathbf{c}$ & Add periodic images \\
\hline
Neighbor List & \texttt{build\_neighbor\_list()} & Sort distances $\leq r_{\text{cut}}$, grouped by type & Identify local atomic environments \\
\hline
\end{tabular}
\end{center}

\end{document}
