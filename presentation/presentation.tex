\documentclass[aspectratio=169]{beamer}
\usetheme{Madrid}
\usecolortheme{default}

\usepackage{amsmath}
\usepackage{amssymb}
\usepackage{graphicx}
\usepackage{physics}
\usepackage{booktabs}
\usepackage{bm}

\title[DeepMD–HALMD Integration]{Integration of Deep Potential Molecular Dynamics (DeepMD v2) into HALMD for Multi-Species Alloy Systems}
\author{Sandip Kumar Sah}
\institute[Universidade de Lisboa]{M.Sc. Computational Science, Freie Universität Berlin }
\date{\today}

\begin{document}

%=====================================================
% \begin{frame}
% \titlepage
% \end{frame}
\begin{frame}
  \titlepage
  \vspace{1cm}
  \begin{center}
    \large \textbf{Supervisor:} Prof. Dr. Felix Höfling
  \end{center}
\end{frame}

%=====================================================
\begin{frame}{Outline}
\tableofcontents
\end{frame}

%=====================================================
\section{Motivation}

\begin{frame}{Motivation}
\begin{itemize}
    \item Accurate interatomic potentials are crucial for Molecular Dynamics (MD).
    \item Classical MD uses empirical potentials (LJ, EAM, etc.) – efficient but approximate.
    \item \textbf{Ab initio MD (AIMD)} provides accuracy but is computationally expensive.
    \item \textbf{Deep Neural Network (DNN) potentials} such as DeepMD bridge this gap:
    \begin{itemize}
        \item Learn potential energy surface (PES) from DFT data.
        \item Achieve near-AIMD accuracy at MD speed.
    \end{itemize}
    \item Goal: Integrate DeepMD v2 potential inference into \textbf{HALMD} for efficient GPU-based alloy simulations.
\end{itemize}
\end{frame}

%=====================================================
\section{Objectives and Scope}

\begin{frame}{Research Objectives}
\begin{itemize}
    \item Integrate DeepMD v2 potential into HALMD’s many-body potential module.
    \item Reproduce energy and force predictions directly from extracted model parameters (\texttt{frozen\_model.pb}).
    \item Support \textbf{multi-species alloy systems} with per-type embedding and fitting networks.
    \item Maintain computational efficiency suitable for GPU-accelerated MD.
\end{itemize}
\end{frame}

%=====================================================
\section{Background}

\begin{frame}{HALMD Software}
\begin{itemize}
    \item \textbf{HALMD (Highly Accelerated Large-scale Molecular Dynamics):}
    \begin{itemize}
        \item Modular C++/CUDA architecture.
        \item Built-in GPU parallelism and neighbor list handling.
        \item Flexible interface for two-body and many-body potentials.
    \end{itemize}
    \item Provides ideal infrastructure to embed neural-network potentials.
\end{itemize}
\end{frame}

\begin{frame}{Deep Potential Molecular Dynamics (DeepMD v2)}
\begin{itemize}
    \item DeepMD decomposes total energy into atomic contributions:
    \[
    E = \sum_i E_i(\mathcal{R}_i)
    \]
    \item Each $E_i$ is predicted by a DNN mapping atomic environments $\mathcal{R}_i$ to energy.
    \item Two main components:
    \begin{enumerate}
        \item \textbf{Filter (Embedding) Network} – encodes local atomic environment.
        \item \textbf{Fitting Network} – maps descriptors to atomic energies.
    \end{enumerate}
    \item Supports \textbf{multi-type systems} with distinct networks per (center, neighbor) pair.
\end{itemize}
\end{frame}

%=====================================================
\section{Methodology Overview}

\begin{frame}{Overview of Implementation}
\begin{center}
% \includegraphics[width=0.9\linewidth]{example-image-a}
\end{center}
\begin{itemize}
    \item Workflow between HALMD and DeepMD:
    \begin{enumerate}
        \item Extract model weights and statistics from \texttt{frozen\_model.pb}.
        \item Reproduce inference pipeline in C++.
        \item Compute per-atom energies and forces in HALMD simulation loop.
    \end{enumerate}
\end{itemize}
\end{frame}

%=====================================================
\section{Stage 1: Coordinate Normalization and Neighbor List}

\begin{frame}{Normalization and Neighbor List Creation}
\begin{itemize}
    \item Wrap atomic coordinates into the primary periodic cell:
    \[
    \mathbf{r}'_i = \mathbf{H}(\mathbf{H}^{-1}\mathbf{r}_i - \lfloor \mathbf{H}^{-1}\mathbf{r}_i \rfloor)
    \]
    \item Create neighbor lists within cutoff radius $r_c$.
    \item Ensures consistent local environments under periodic boundaries.
    \item Step is essential before descriptor computation.
\end{itemize}
\begin{center}
% \includegraphics[width=0.7\linewidth]{example-image-b}
\end{center}
\end{frame}

%=====================================================
\section{Stage 2: Descriptor Formation}

\begin{frame}{Environment Matrix and Descriptor Construction}
\begin{itemize}
    \item Compute environment matrix $\mathbf{R}$ for each atom:
    \[
    \mathbf{R}_{ij} =
    s(r_{ij})
    \begin{bmatrix}
        1/r_{ij} & r_{ij,x}/r_{ij}^2 & r_{ij,y}/r_{ij}^2 & r_{ij,z}/r_{ij}^2
    \end{bmatrix}
    \]
    \item Apply smooth cutoff $s(r)$ to ensure differentiability.
    \item Normalize $\mathbf{R}$ using species-specific mean and std.
    \item Pass scalar component $\hat{s}_{ij} = 1/r_{ij}$ through \textbf{filter networks}.
\end{itemize}
\end{frame}

\begin{frame}{Descriptor Matrix Formation}
\begin{itemize}
    \item Embedding output $\mathbf{G} \in \mathbb{R}^{N_{\text{neigh}} \times M_1}$.
    \item Descriptor in HALMD (explicit formulation):
    \[
    \mathbf{D}^{(i)} = \frac{1}{N_{\text{neigh}}^2}
    \mathbf{G}^{\top} (\mathbf{R}\mathbf{R}^{\top}) \mathbf{G}_{<}
    \]
    \item Flattened vector $\mathcal{D}_i = \mathrm{vec}(\mathbf{D}^{(i)})$ is used for energy prediction.
    \item HALMD performs these operations with \texttt{boost::ublas} for clarity and derivative consistency.
\end{itemize}
\end{frame}

%=====================================================
\section{Stage 3: Fitting Network and Energy Evaluation}

\begin{frame}{Fitting Network Structure}
\begin{itemize}
    \item Each atom type $t_c$ has its own fitting MLP.
    \item Hidden layers:
    \[
    \mathbf{x}^{(l+1)} =
    \begin{cases}
        \mathbf{x}^{(l)} + \texttt{idt}^{(l)} \odot \tanh(\mathbf{x}^{(l)}\mathbf{W}^{(l)}+\mathbf{b}^{(l)}), & \text{if residual present}, \\[5pt]
        \tanh(\mathbf{x}^{(l)}\mathbf{W}^{(l)}+\mathbf{b}^{(l)}), & \text{otherwise.}
    \end{cases}
    \]
    \item Final output (linear layer):
    \[
    E_i = \mathbf{x}^{(L)}\mathbf{W}^{(\text{final})} + \mathbf{b}^{(\text{final})} + b_{t_c(i)}
    \]
    \item Total potential energy:
    \[
    E_{\text{total}} = \sum_i E_i
    \]
\end{itemize}
\end{frame}

%=====================================================
\section{Results and Discussion}

\begin{frame}{Implementation Outcomes}
\begin{itemize}
    \item Full inference pipeline implemented in HALMD (C++).
    \item Supports multiple atomic species (e.g., Cu–Ag alloys).
    \item Accurately reproduces DeepMD energies and forces.
    \item Maintains computational efficiency with GPU parallelism.
\end{itemize}
\begin{center}
% \includegraphics[width=0.8\linewidth]{example-image-c}
\end{center}
\end{frame}

%=====================================================
\section{Conclusion}

\begin{frame}{Conclusion}
\begin{itemize}
    \item Successfully integrated DeepMD v2 inference into HALMD.
    \item Achieved energy and force consistency across Python and C++ implementations.
    \item Extended HALMD to handle alloy systems with multi-type neural potentials.
    \item Lays foundation for future work:
    \begin{itemize}
        \item Gradient-based force computation.
        \item Fully GPU-optimized neural inference.
        \item Integration with HALMD’s thermostat and ensemble modules.
    \end{itemize}
\end{itemize}
\end{frame}

%=====================================================
\begin{frame}
\centering
\Huge Thank You!
\end{frame}

\end{document}
